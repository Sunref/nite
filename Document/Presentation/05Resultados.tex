\section{Resultados}

\begin{frame}

    \frametitle{Planejamento e Resultados}

    Idealizando o projeto como um todo e utilizando a ferramenta Obsidian para notas, foi possível compreender melhor o projeto e definir um planejamento..

\end{frame}

\begin{frame}

    \frametitle{Interface Inicial}

    \begin{figure}[!htbp]
       	\centering
       	\includegraphics[scale=0.2]{imagens/nite.png}
    \end{figure}

\end{frame}

\begin{frame}

	\frametitle{Funcionalidades}
	\scriptsize
	\begin{table}[h]
		\centering
		\caption{Extensões aceitas e bloqueadas pelo editor NITE}
		\renewcommand{\arraystretch}{1.1}
		\setlength{\tabcolsep}{4pt}
			\begin{tabular}{@{}p{2.8cm}p{4.8cm}p{4.8cm}@{}}
			\toprule
			\textbf{Categoria} & \textbf{Extensões válidas} & \textbf{Extensões não válidas} \\
			\midrule
			Texto plano & .txt, .md, .csv, .log & – \\
			Código-fonte & Arquivos de texto contendo código (.c, .cpp, .java, .py, .js, .html, .css, ...) & – \\
			Configurações & .ini, .conf, .properties, .toml & – \\
			Scripts executáveis & .sh, .bat, .ps1, .cmd & – \\
			Executáveis/Binários & – & .exe, .dll, .so, .bin, .out, .app \\
			Mídia & – & .png, .jpg, .gif, .mp3, .mp4, ... \\
			Compactados & – & .zip, .rar, .tar, .gz, .7z \\
			Documentos binários & – & .doc, .pptx, .xlsx, .pdf \\
			Imagens de disco & – & .iso, .img \\
			\bottomrule
			\end{tabular}
	\end{table}

\end{frame}