\section{Metodologia}


\begin{frame}{Metodologia}

% ------------------------
% PALETA DE VERDES
% ------------------------
\definecolor{verdeClaro}{RGB}{199, 233, 204}
\definecolor{verdeMedio}{RGB}{161, 217, 155}
\definecolor{verdeForte}{RGB}{116, 196, 118}
\definecolor{verdeEscuro}{RGB}{49, 163, 84}

% Estilo das caixas
\tikzset{
    box/.style={
        rectangle,
        rounded corners=8pt,
        text width=3.2cm,
        align=center,
        minimum height=1.2cm,
        font=\bfseries,
        drop shadow
    }
}

\centering
\begin{tikzpicture}[node distance=0.8cm]

% TÍTULO NITE — centralizado
\node[
    rectangle,
    rounded corners=10pt,
    minimum width=4.5cm,
    minimum height=1.2cm,
    fill=verdeForte,
    text=white,
    font=\Large\bfseries,
    drop shadow
] (title) at (0, 2) {NITE};

% ------------------------------------------------
% PRIMEIRA LINHA — agora com mais espaço lateral
% ------------------------------------------------
\node[box, fill=verdeClaro]       (n1) at (-5.6, 0.7) {1. Criar protótipo e modos};
\node[box, fill=verdeMedio]       (n2) at (-1.9, 0.7) {2. Criar e abrir arquivos};
\node[box, fill=verdeForte]       (n3) at (1.9, 0.7)  {3. Editar e salvar};
\node[box, fill=verdeEscuro!85]   (n4) at (5.6, 0.7)  {4. Múltiplos arquivos};

% ------------------------------------------------
% SEGUNDA LINHA — mesma separação aumentada
% ------------------------------------------------
\node[box, fill=verdeClaro]       (n5) at (-5.6, -0.9) {5. Atalhos\\\textit{shortcuts}};
\node[box, fill=verdeMedio]       (n6) at (-1.9, -0.9) {6. Ajustes visuais};
\node[box, fill=verdeForte]       (n7) at (1.9, -0.9)  {7. \textit{Syntax highlight}};
\node[box, fill=verdeEscuro!85]   (n8) at (5.6, -0.9)  {8. Novas funcionalidades};

\end{tikzpicture}

\end{frame}


\begin{frame}

    \frametitle{Metodologia}

    \begin{itemize}
    	\item Ferramentas utilizadas:
     	\begin{itemize}
    	 	\item Linguagem de programação C utilizada no editor de texto Zed; Biblioteca ncurses para C, GCC como compilador e terminal BlackBox para testes; Obsidian para gerenciamento de notas; Desenvolvido em uma máquina Linux
        \end{itemize}
        \item Detalhes do desenvolvimento
        \begin{itemize}
      	 	\item Análise das funcionalidades vista nos editores citados anteriormente foram implementadas conforme o objetivo final do NITE. Experiência de uso das ferramentas correlatas ao projeto conta como fator importante para o sucesso das implementações.
          	\end{itemize}
    \end{itemize}

\end{frame}