\chapter{Metodologia}
\label{cap:03}

%Descrever metodologia, materiais e métodos utilizados no estudo, bem como os procedimentos experimentais realizados (equipamentos, técnicas e processos utilizados).

Idealmente planejado como um projeto incremental, o NITE segue a linha de ``entregas'' em bloco,
implementações lentas e funcionais seguindo a ordem descrita na tabela abaixo.

\FloatBarrier
\begin{table}[!htbp]
	\centering
	\begin{tabular}{ c | c }
		\hline
		\textbf{Ordem} & \textbf{Implementação}                                     \\ \hline
		1  & Criar protótipo inicial                                                \\ \hline
		2  & Implementar criação e abertura de arquivos                             \\ \hline
		3  & Implementar edição e salvamento de arquivos                            \\ \hline
		4  & Implementar suporte a múltiplos arquivos                               \\ \hline
		5  & Realizar ajustes visuais e de interface                                \\ \hline
		6  & Implementar atalhos de teclado / \textit{shortcuts}                    \\ \hline
		7  & Implementar \textit{syntax highlight}\footnotemark                     \\ \hline
		8  & Buscar novas funcionalidades                                           \\ \hline
	\end{tabular}
	\\ \vspace{0.2cm}
	\textbf{Fonte:} Elaborada pelo autor
	\label{tab:implementacoes}
\end{table}
\footnotetext{Destaque de sintaxe}
\FloatBarrier

Isso permite testar as funcionalidades implementadas cuidadosamente, e realizar ajustes
sempre que necessário de forma ágil e controlada.

Descrito na sessão anterior, a linguagem de desenvolvimento que mais se adequa a proposta do projeto
foi a linguagem C, por sua confiabilidade e utilização continua em projetos semelhantes. Nessa lógica
a biblioteca utilizada para manipulação de terminal foi a Ncurses, muito bem fundamentada e completa
de funcionaidades que serão aproveitadas ao máximo. Como ferramentas auxiliares, o uso do Git para
versionamento de arquivos, o editor Zed como IDE principal de desenvolvimento e o ambiente Linux para
testes. Vale ressaltar aqui o uso do GCC\footnote{Coleção de compiladores do projeto GNU} como compilador.

