\chapter{Metodologia}
\label{cap:03}

Idealmente planejado como um projeto incremental, o NITE segue a linha de desenvolvimento
por partes, com implementações lentas e funcionais, como sugere o diagrama abaixo:

\begin{center}
    \begin{tikzpicture}[scale=1.1, transform shape]
        \path[
            mindmap,
            concept color=blue!30,
            text=black,
            level 1/.append style={sibling angle=45, level distance=4.5cm},
            every node/.style={align=center}
        ]
            node[concept, font=\Large\bfseries, minimum size=2.5cm] {NITE}
            [clockwise from=0]
            child[concept color=green!40, grow=0]
            { node[concept, font=\footnotesize, minimum size=2.8cm] {5. Implementar\\atalhos de\\teclado e modos/\\\textit{shortcuts}}}
            child[concept color=orange!40, grow=45]
            { node[concept, font=\footnotesize, minimum size=3.2cm] {4. Implementar\\suporte a\\múltiplos\\arquivos}}
            child[concept color=yellow!70, grow=90]
            { node[concept, font=\footnotesize, minimum size=3.2cm] {3. Implementar\\edição e\\salvamento\\de arquivos} }
            child[concept color=purple!40, grow=135]
            { node[concept, font=\footnotesize, minimum size=3cm] {2. Implementar\\criação e\\abertura de\\arquivos}}
            child[concept color=pink!60, grow=180]
            { node[concept, font=\footnotesize, minimum size=3.2cm] {1. Criar\\protótipo\\inicial e ideia de modos}}
            child[concept color=teal!40, grow=225]
            { node[concept, font=\footnotesize, minimum size=3.5cm] {8. Buscar\\novas\\funcionalidades}}
            child[concept color=red!40, grow=270]
            { node[concept, font=\footnotesize, minimum size=3cm] {7. Implementar\\\textit{syntax}\\\textit{highlight}\footnotemark[1]} }
            child[concept color=lime!60, grow=315]
            { node[concept, font=\footnotesize, minimum size=3cm] {6. Realizar\\ajustes visuais\\e de interface}};
    \end{tikzpicture}
    \footnotetext[1]{Destaque de sintaxe}
\end{center}

Explicando, detalhadamente, cada passo:

\begin{itemize}
    \item \textbf{Primeiro passo:} O primeiro passo consiste na criação de um protótipo simples. Idealmente, o protótipo inicial
    pode se basear na interface do Vim e seus derivados, com uma tela de início e atalhos uteis a disposição. Além de um protótipo simples,
    o NITE também será desenvolvido com base em dois modos: edição e leitura. Esses modos devem ser analisados e implementados posteriormente.
    \item \textbf{Segundo passo:} O segundo passo é a implementação de abrir e criar arquivos atravez do NITE, o objetivo aqui é implementar
    apenas esses passos, inicialmente para arquivos de formato *.txt.
    \item \textbf{Terceiro passo:} O tercerceito passo, após a implementação do segundo, é garantir que os arquivos possam ser editados e salvos.
    Aqui, a ideia inicial e definida no primeiro passo de dois modos de uso deve ser implementado e testado.
    \item \textbf{Quarto passo:} Seguindo, o quarto passo complementa o segundo. Aqui deve ser implementado suporte a diversos tipos de
    arquivos, e de forma sucinta abrir e criar esses arquivos.
    \item \textbf{Quinto passo:} O quinto passo tem como objetivo criar e implementar atalhos de teclados personalizados para o NITE, baseando em
    atalhos do Vim (e seus derivados), Nano e Emacs (e seus derivados).
    \item \textbf{Sexto passo:} O sexto passo foca em realizar ajustes visuais e dar forma ao projeto, trazendo identidade visual. Essa decisão se dá pois o foco principal do projeto é funcionalidade, e assim com o tempo aplicar caracteristicas visuais unicas.
    \item \textbf{Sétimo passo:} Com o sistema parcialmente pronto, o sétimo passo busca implementar destaque de sintaxe para diferentes
    linguagens de programação. Nesta etapa, a biblioteca Tree-Sitter \cite{tree-sitter} será estudada e utilizada, gerando árvores de sintaxe abstratas que ajudarão na implementação de destaques para diversos tipos de linguagens implementadas para o projeto.
    \item \textbf{Oitavo passo:} Por fim, o oitavo passo aponta para futuros estudos e avanços do projeto como um todo, onde novas funcionalidades
    serão buscadas e estudadas para versões futuras do NITE.
\end{itemize}

Consequentemente, isso permite testes cuidadosos das funcionalidades implementadas,
além de possibilitar ajustes sempre que necessário de forma ágil e controlada.

Descrito na sessão anterior, a linguagem de desenvolvimento que mais se adequa a
proposta do projeto foi a linguagem C, por sua confiabilidade e utilização
continua em projetos semelhantes. Nessa lógica a biblioteca utilizada para manipulação
de terminal foi a Ncurses, muito bem fundamentada e completa de funcionaidades
que serão aproveitadas ao máximo. Como ferramentas auxiliares, o uso do Git para
versionamento de arquivos, o editor Zed como IDE principal de desenvolvimento complementar ao ambiente
Linux com o terminal BlackBox para testes. Vale ressaltar aqui o uso do
GCC\footnote{Coleção de compiladores do projeto GNU} como compilador.

Cada etapa do desenvolvimento é realizada em paralelo a análises e estudos do código-fonte
de outros editores, como o NeoVim, com ênfase no código do Nano, editor que
serviu de base para o NITE.