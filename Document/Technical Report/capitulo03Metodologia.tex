\chapter{Metodologia}
\label{cap:03}

Idealmente planejado como um projeto incremental, o NITE segue a linha de ``entregas''
por partes, com implementações lentas e funcionais, mostrado pelo diagrama abaixo:

\begin{center}
    \begin{tikzpicture}[scale=1.1, transform shape]
        \path[
            mindmap,
            concept color=blue!30,
            text=black,
            level 1/.append style={sibling angle=45, level distance=4.5cm},
            every node/.style={align=center}
        ]
            node[concept, font=\Large\bfseries, minimum size=2.5cm] {NITE}
            [clockwise from=0]
            child[concept color=green!40, grow=0]
            { node[concept, font=\footnotesize, minimum size=2.8cm] {1. Criar\\protótipo\\inicial} }
            child[concept color=orange!40, grow=45]
            { node[concept, font=\footnotesize, minimum size=3.2cm] {2. Implementar\\criação e\\abertura de\\arquivos} }
            child[concept color=yellow!70, grow=90]
            { node[concept, font=\footnotesize, minimum size=3.2cm] {3. Implementar\\edição e\\salvamento\\de arquivos} }
            child[concept color=purple!40, grow=135]
            { node[concept, font=\footnotesize, minimum size=3cm] {4. Implementar\\suporte a\\múltiplos\\arquivos} }
            child[concept color=pink!60, grow=180]
            { node[concept, font=\footnotesize, minimum size=3.2cm] {5. Realizar\\ajustes visuais\\e de interface} }
            child[concept color=teal!40, grow=225]
            { node[concept, font=\footnotesize, minimum size=3.5cm] {6. Implementar\\atalhos de\\teclado /\\\textit{shortcuts}} }
            child[concept color=red!40, grow=270]
            { node[concept, font=\footnotesize, minimum size=3cm] {7. Implementar\\\textit{syntax}\\\textit{highlight}\footnotemark[1]} }
            child[concept color=lime!60, grow=315]
            { node[concept, font=\footnotesize, minimum size=3cm] {8. Buscar\\novas\\funcionalidades} };
    \end{tikzpicture}
    \footnotetext[1]{Destaque de sintaxe}
\end{center}

Consequentemente, isso permite testes cuidadosos das funcionalidades implementadas,
além de possibilitar ajustes sempre que necessário de forma ágil e controlada.

Descrito na sessão anterior, a linguagem de desenvolvimento que mais se adequa a
proposta do projeto foi a linguagem C, por sua confiabilidade e utilização
continua em projetos semelhantes. Nessa lógica a biblioteca utilizada para manipulação
de terminal foi a Ncurses, muito bem fundamentada e completa de funcionaidades
que serão aproveitadas ao máximo. Como ferramentas auxiliares, o uso do Git para
versionamento de arquivos, o editor Zed como IDE principal de desenvolvimento e o
ambiente Linux com o terminal BlackBox para testes. Vale ressaltar aqui o uso do
GCC\footnote{Coleção de compiladores do projeto GNU} como compilador.

Cada etapa do desenvolvimento é realizada em paralelo a análises e estudos do código-fonte
de outros editores, como o NeoVim, com ênfase no código do Nano, editor que
serviu de base para o NITE. Já na implementação de destaque de sintaxe, a
biblioteca Tree-Sitter será estudada e utilizada, gerando árvores de sintaxe abstratas
que ajudarão na implementação de destaque de sintaxe para o projeto \cite{tree-sitter}.