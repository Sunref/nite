\chapter{Introdução}
\label{cap:01}

O constante avanço tecnológico impulsiona o desenvolvimento de novas ferramentas e técnicas para atender às demandas da programação.
Diante da diversidade de soluções e editores de texto, os desenvolvedores buscam ferramentas que se adaptem melhor às suas necessidades.
O NITE surge como uma proposta nesse cenário, um editor de linha de comando simples, compacto e poderoso para o desenvolvimento de
aplicações. Inspirado em conceitos modernos e nas funcionalidades de ferramentas predecessoras, busca combinar eficiência e praticidade.
Este trabalho apresenta o estudo de editores e bibliotecas correlatas ao NITE, analisando suas funcionalidades, vantagens e aplicações
práticas, com o objetivo de compreender como elas contribuem para a implementação e evolução de sistemas de linha de comando.

\section{Objetivos}

\subsection{Objetivo Geral}

Analisar e compreender as ferramentas e bibliotecas correlatas ao NITE, investigando suas funcionalidades, vantagens e aplicações
práticas no desenvolvimento de sistemas de linha de comando.

\subsection{Objetivos Específicos}
\begin{itemize}
    \item Mapear e descrever ferramentas e bibliotecas relacionadas a edição de texto em terminais, destacando suas principais
    características.
    \item Identificar boas práticas e conceitos modernos aplicados na implementação dessas ferramentas.
    \item Avaliar como essas ferramentas podem contribuir para a evolução e melhoria do NITE.
    \item Desenvolver a aplicação final baseada nos estudos anteriores.
\end{itemize}
