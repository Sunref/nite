\chapter{Considerações Gerais}
\label{cap:02}

Texto das considerações gerais, dividido em subseções.


% Este é um exemplo de como usar figuras. Referência cruzada: Figura~\ref{fig:exemplo}

% \FloatBarrier
% \begin{figure}[!htbp]
% 	\centering
% 	\caption{Exemplo de figura}
% 	%scale redimensiona a figura.
% 	%1.5 = 150% do tamanho original
% 	%1 = 100% do tamanho original
% 	%0.20 = 20% do tamanho original
% 	\includegraphics[scale=0.4]{imagens/exemploFigura}
% 	\\\textbf{Fonte:} Elaborada pelo autor
% 	\label{fig:exemplo}
% \end{figure}
% \FloatBarrier


% Este é um exemplo de como usar equações. Referência cruzada: Equação~\ref{eq:exemplo}

% \begin{equation}
% \sum_{i=1}^{n} i = \frac{n(n+1)}{2}
% \label{eq:exemplo}
% \end{equation}
%
%

% Exemplo de inserção de lista de código fonte:

% \lstinputlisting[language=Java]{fontes/ClasseExemplo.java}


% Este é um exemplo de como inserir texto sem formatação (ambiente verbatim):

% \begin{verbatim}
% Texto sem formatação, como espaçamento igual.
% \end{verbatim}


% Exemplos de comandos para texto e referências:

% \begin{itemize}
% 	\item Para iniciar um novo parágrafo, basta deixar uma linha em branco no código fonte;
% 	\item Não force o compilador a pular mais de uma linha, pois terá influência negativa na composição do documento;
% 	\item Sempre deixe o \LaTeX\ realizar a formatação de parágrafos e posicionamento de elementos;
% 	\item Utilização de aspas simples (abertura \verb|`|, fechamento \verb|'|): `Texto entre aspas simples';
% 	\item Utilização de aspas duplas (abertura \verb|``|, fechamento \verb|''|): ``Texto entre aspas duplas'';
% 	\item Negrito (comando \verb|\textbf|): \textbf{texto em negrito};
% 	\item Itálico (comando \verb|\textit|): \textit{texto em itálico};
% 	\item Sublinhado (comando \verb|\underline|): \underline{texto sublinhado};
% 	\item Negrito e itálico (usar comandos juntos): \textbf{\textit{texto em negrito e itálico}};
% 	\item Alterar cor do texto (comando \verb|\textcolor{cor}{texto}|):
% 	\begin{itemize}
% 		\item Exemplo \verb|\textcolor{red}{texto}|: \textcolor{red}{texto vermelho};
% 		\item Exemplo \verb|\textcolor[RGB]{255, 102, 0}|: \textcolor[RGB]{255, 102, 0}{texto laranja};
% 		\item Exemplo \verb|\textcolor[HTML]{006AD7}|: \textcolor[HTML]{006AD7}{texto azul};
% 	\end{itemize}
% 	\item Ambiente matemático inline (comando \verb|$ expressão $|): $s = x^2-2x +1$;
% 	\item Referência normal (comando \verb|\cite|):
% 	\begin{itemize}
% 		\item \cite{Agaisse1995};
% 		\item \cite{Abedi2014};
% 		\item \cite{BtNomenclature2016};
% 	\end{itemize}
% 	\item Referência normal com mais de uma obra (comando \verb|\cite|):
% 	\begin{itemize}
% 		\item \cite{Abedi2014, Agaisse1995};
%         \item \cite{AgapitoTenfen2014, BtNomenclature2016, Nelson2014};
% 	\end{itemize}
% 	\item Referência nome e ano (comando \verb|\citeauthorandyear|):
% 	\begin{itemize}
% 		\item \citeauthorandyear{Agaisse1995};
% 		\item \citeauthorandyear{Abedi2014};
% 		\item \citeauthorandyear{BtNomenclature2016};
% 	\end{itemize}
% \end{itemize}


% Exemplo 1 de citação direta:

% \begin{citacao}
% 	Os 20 aminoácidos usualmente encontrados como resíduos em proteínas contém um grupo $\alpha$-carboxil, um grupo $\alpha$-amino e um grupo R
% 	distinto substituído no átomo de carbono $\alpha$. O átomo de carbono $\alpha$ de todos os aminoácidos, com exceção da glicina, é assimétrico e,
% 	portanto, os aminoácidos podem existir em pelo menos duas formas estereoisoméricas. Somente os estereoisômeros L, com uma configuração relacionada à
% 	configuração absoluta da molécula de referência L-gliceraldeído, são encontrados em proteínas \cite[p. 81]{Nelson2014}.
% \end{citacao}

% Exemplo 2 de citação direta:

% \begin{citacao}
% 	\textit{These various insecticidal proteins are synthesized during the stationary phase and accumulate in the mother cell as a crystal inclusion
% 	which can account for up to 25\% of the dry weight of the sporulated cells. The amount of crystal protein produced by a B. thuringiensis culture in
% 	laboratory conditions (about 0.5 mg of protein per ml) and the size of the crystals (24) indicate that each cell has to synthesiz
% 	e $10^6$ to $2 \times 10^6$ $\delta$-endotoxin molecules during the stationary phase to form a crystal} \cite[p. 1]{Agaisse1995}.
% \end{citacao}

% Exemplo de nota de rodapé\footnote{Essa é uma nota de rodapé!}.

\section{Trabalhos Correlatos}

Gophertype: Criação de um editor/aplicativo de treino de digitação em modo texto (TUI) usando Go + Bubble Tea framework. \cite{Lindroos2025}

An Empirical Investigation of Command-Line Customization: Analisa mais de 2,2 milhões de aliases no GitHub e identifica padrões de personalização do shell:
atalhos, modificações e scripts \cite{SchroderCito2022}

\section{Interface Modo Texto x Modo Gráfico}

Como os usuários interagiam com computadores antes das interfaces gráficas? Nos primórdios da computação, programas e dados eram armazenados em cartões
perfurados, estes que eram pré-processados em lote e com longa espera de resultados. Com a constante evolução das maquinas e a criação dos mainframes,
grandes e poderosos computadores que suportavam diversas interações simultaneamente e com níveis de confiabilidade e multitarefa, fez com que a comunicação
entre usuário e máquina passasse a ser realizada de forma mais iterativa, principalmente por meio de interfaces de linha de comando (CLI),
como o TTY (Teletypewriter) \cite{ColumbiaTeletype2023} e o terminal VT100 \cite{DEC_VT100}. Nestes sistemas os usuários digitavam comandos diretamente em um
terminal e recebiam uma saída quase em tempo real \cite{ComputerHistoryMuseum}.

Diferente da atualidade dominada pelas interfaces gráficas (GUI), as interfaces de linha de comando exigiam que os usuários conhecessem os comandos exatos
para realizar tarefas simples, como copiar arquivos ou executar programas. Essa ausência de uma interface gráfica significava que os usuários precisavam
memorizar uma série de comandos e sintaxes, o que tornava o uso do computador muito mais técnico e menos acessível ao público em geral.

O terminal VT100 foi um marco por introduzir códigos de escape ANSI (American National Standards Institute) e permitir controle de cursor e formatação de texto,
características que tornaram a linha de comando mais funcional e eficiente. Apesar disso, o uso da CLI continuava exigindo conhecimento técnico avançado, o que
ainda limitava bastante o acesso ao uso do computador.

Mas os avanços não se limitaram apenas ao hardware, as constantes descobertas na tecnologia possibilitaram a criação dos primeiros Sistemas Operacionais (OS),
como o UNIX \cite{UnixArchive} e o MS-DOS \cite{ComputerHistoryMuseum}, popularizando o uso de terminais e CLIs e os tornando mais acessíveis ao público. Ainda
no início do UNIX surgiram os primeiros editores, como o Vi \cite{Joy_Vi} e o Emacs \cite{Stallman1981}, possibilitando um uso mais robusto da máquina com
execução de comandos em tempo real, automatização de tarefas e o principal, a manipulação de texto.

Baseado no editor EX, o Vi foi criado e rapidamente se tornou um dos editores de texto mais icônicos e amplamente utilizados no mundo UNIX.
O Vi introduziu uma abordagem modal para edição de texto, onde os usuários podiam alternar entre modos de inserção e comando, permitindo uma edição mais
eficiente. O Emacs, por outro lado  se destacou por sua extensibilidade e personalização, permitindo que os usuários criassem seus próprios comandos e macros.
Atualmente existem diversas variações desses editores, como o Vim (Vi Improved) \cite{Vim} e o NeoVim, \cite{Neovim_Project} versões aprimoradas do Vi, e
também com o Emacs, possuíndo diversas variações e extensões, como o Spacemacs \cite{Spacemacs} e o Doom Emacs \cite{DoomEmacs}, todos seguindo a mesma
filosofia de edição de texto em CLI.

Já na linha do MS-DOS, o editor de texto mais conhecido é o EDIT.COM \cite{MicrosoftEdit2025}, que foi introduzido no MS-DOS 2.0 em 1983. O EDIT.COM era um
editor de texto simples, mas permitia aos usuários criar e editar arquivos de texto diretamente no terminal. Com o tempo, outros editores de texto CLI surgiram,
como o Nano \cite{Nano2025}, que se destacou por sua simplicidade e facilidade de uso, tornando-se uma escolha popular para usuários que preferiam uma interfac
e mais amigável em comparação com o Vi e o Emacs. O Nano foi projetado para ser um editor de texto fácil de usar, com uma interface intuitiva e comandos simples,
tornando-o acessível para usuários menos experientes. Recentemente a Microsoft relançou o editor EDIT como parte do Windows Terminal, mantendo a mesma
simplicidade e funcionalidade do original, mas com melhorias na interface e na integração com o Windows.

\FloatBarrier
\begin{figure}[!htbp]
	\centering
	\caption{EDIT.COM no MS-DOS}
	\includegraphics[scale=0.8]{imagens/EDIT_MSDOS}
	\\\textbf{Fonte: Internet Archive}
	\label{fig:EDIT_MSDOS}
\end{figure}
\FloatBarrier

Mais adiante na história, com o avanço das interfaces gráficas os editores de texto passaram a se tornar mais visuais. É possível notar essa evolução em
softwares como o Notepad++ \cite{NotepadPlusPlus2025} até editores mais famosos como Visual Studio Code \cite{VSCode2025} e o recente Zed
\cite{ZedEditor2025}. Esses editores de texto modernos oferecem uma ampla gama de recursos, como customização e extensibilidade, suporte a plugins,
integração com sistemas de controle de versão, destacamento de sintaxe, autocompletar, entre outros, tornando a edição de texto mais eficiente e produtiva.

O Notepad++ é um editor de texto de código aberto que se destaca por sua leveza e simplicidade, oferecendo uma interface amigável e recursos avançados
para edição de código-fonte. Ele suporta várias linguagens de programação e possui recursos como realceamento de sintaxe, autocompletar, busca avançada e
suporte a plugins, tornando-o uma escolha popular entre usuários que buscam um editor de texto simples mas eficiente.

Por outro lado o Visual Studio Code é um editor de código-fonte desenvolvido pela Microsoft que se tornou extremamente popular devido à sua extensibilidade e
integração com diversas ferramentas de desenvolvimento, além de oferecer uma ampla gama de recursos como depuração integrada, controle de versão, suporte a
várias linguagens de programação e uma vasta biblioteca de extensões, permitindo que os usuários personalizem sua experiência de edição de texto de acordo com
suas necessidades.

\FloatBarrier

\begin{figure}[!htbp]
	\centering
	\caption{Visual Studio Code}
	\includegraphics[scale=0.3]{imagens/VSCode}
	\\\textbf{Fonte: Github Microsoft - Visual Studio Code}
	\label{fig:VSCode}
\end{figure}
\FloatBarrier

Já o ainda em desenvolvimento Zed Editor se destaca por sua abordagem colaborativa e em tempo real, permitindo que vários usuários editem o mesmo
arquivo simultaneamente. Ele oferece uma interface moderna e intuitiva e que se assemelha muito ao Visual Studio Code, com recursos como realceamento de sintaxe,
autocompletar, integração com sistemas de controle de versão e suporte a plugins, tornando-o uma escolha interessante para equipes de desenvolvimento que buscam
uma experiência de edição colaborativa e eficiente.

Ainda hoje, após décadas de evolução, há usuários que preferem interagir por meio de CLIs, especialmente em ambientes de desenvolvimento, como ao utilizar o
editor NeoVim em sistemas UNIX. Com base nisso, este trabalho propõe a criação de um editor CLI inspirado no Nano, mas com uma interface mais moderna e
intuitiva, mantendo a essência dos clássicos editores de texto em terminal mencionados anteriormente.

\section{Editores, Processadores de Texto e IDEs}

MEIA PÁGINA PARA CADA UM (ou mais se julgar necessário)

\subsection{Editores}

EDITORES: Vi, Emacs, Nano, NeoVim, Visual Studio Code, Zed, Notepad++, etc. Falar um pouco mais sobre como eles funcionam (já que já foram citados).

\subsection{Processadores de Texto}
PROCESSADORES DE TEXTO: Microsoft Word, LibreOffice Writer, Google Docs, etc.

\subsection{IDEs}

IDEs: Visual Studio, CodeBlocks, NetBeans, IntelliJ IDEA, PyCharm, etc.

\section{Principais Bibliotecas para Controle de Terminal em Linguagem C}

MEIA PÁGINA PARA CADA BIBLIOTECA

Falar um pouco sobre cada uma e por que a escolhida foi a ncurses.

NCURSES e CURSES:

TERMBOX:

TERMINFO e TERMCAP:

S-LANG:
