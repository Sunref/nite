\chapter{Considerações Gerais}
\label{cap:02}

Texto das considerações gerais, dividido em subseções. % TODO: Perguntar ao professor

% Este é um exemplo de como usar figuras. Referência cruzada: Figura~\ref{fig:exemplo}

% \FloatBarrier
% \begin{figure}[!htbp]
% 	\centering
% 	\caption{Exemplo de figura}
% 	%scale redimensiona a figura.
% 	%1.5 = 150% do tamanho original
% 	%1 = 100% do tamanho original
% 	%0.20 = 20% do tamanho original
% 	\includegraphics[scale=0.4]{imagens/exemploFigura}
% 	\\\textbf{Fonte:} Elaborada pelo autor
% 	\label{fig:exemplo}
% \end{figure}
% \FloatBarrier


% Este é um exemplo de como usar tabelas. Referência cruzada: Tabela~\ref{tab:exemplo}

% \FloatBarrier
% \begin{table}[!htbp]
% 	\centering
% 	\caption{Exemplo de tabela de 2 colunas}
% 	\begin{tabular}{ c | c }
% 		\hline
% 		\textbf{Coluna 1} & \textbf{Coluna 2} \\ \hline
% 		Dado 1a           & Dado 2a           \\ \hline
% 		Dado 1b           & Dado 2b           \\ \hline
% 		Dado 1c           & Dado 2c           \\ \hline
% 		Dado 1d           & Dado 2d           \\ \hline
% 	\end{tabular}
% 	\\ \vspace{0.2cm}
% 	\textbf{Fonte:} Elaborada pelo autor
% 	\label{tab:exemplo}
% \end{table}
% \FloatBarrier


% Este é um exemplo de como usar quadros. Referência cruzada: Quadro~\ref{qua:exemplo}

% \FloatBarrier
% \begin{quadro}[!htbp]
% 	\centering
% 	\caption{Exemplo de quadro}
% 	\includegraphics[scale=.7]{imagens/exemploQuadro}
% 	\\\textbf{Fonte:} Elaborada pelo autor
% 	\label{qua:exemplo}
% \end{quadro}
% \FloatBarrier


% Este é um exemplo de como usar equações. Referência cruzada: Equação~\ref{eq:exemplo}

% \begin{equation}
% \sum_{i=1}^{n} i = \frac{n(n+1)}{2}
% \label{eq:exemplo}
% \end{equation}


% Exemplo de inserção de lista de código fonte:

% \lstinputlisting[language=Java]{fontes/ClasseExemplo.java}


% Este é um exemplo de como inserir texto sem formatação (ambiente verbatim):

% \begin{verbatim}
% Texto sem formatação, como espaçamento igual.
% \end{verbatim}


% Exemplo de lista de itens:

% \begin{itemize}
% 	\item \textbf{Item 1:} texto...;
% 	\item \textbf{Item 2:} texto...;
% 	\begin{itemize}
% 		\item \textbf{Subitem:} texto...;
% 		\item \textbf{Subitem:} texto...;
% 		\item \textbf{Subitem:} texto...;
% 	\end{itemize}
% 	\item \textbf{Item 3:} texto...;
% 	\item \textbf{Item n:} texto....
% \end{itemize}


% Exemplo de lista numerada:

% \begin{enumerate}
% 	\item \textbf{Item:} texto...;
% 	\item \textbf{Item:} texto...;
% 	\begin{enumerate}
% 		\item \textbf{Subitem:} texto...;
% 		\item \textbf{Subitem:} texto...;
% 		\item \textbf{Subitem:} texto...;
% 	\end{enumerate}
% 	\item \textbf{Item:} texto...;
% 	\item \textbf{Item:} texto....
% \end{enumerate}


% Exemplos de comandos para texto e referências:

% \begin{itemize}
% 	\item Para iniciar um novo parágrafo, basta deixar uma linha em branco no código fonte;
% 	\item Não force o compilador a pular mais de uma linha, pois terá influência negativa na composição do documento;
% 	\item Sempre deixe o \LaTeX\ realizar a formatação de parágrafos e posicionamento de elementos;
% 	\item Utilização de aspas simples (abertura \verb|`|, fechamento \verb|'|): `Texto entre aspas simples';
% 	\item Utilização de aspas duplas (abertura \verb|``|, fechamento \verb|''|): ``Texto entre aspas duplas'';
% 	\item Negrito (comando \verb|\textbf|): \textbf{texto em negrito};
% 	\item Itálico (comando \verb|\textit|): \textit{texto em itálico};
% 	\item Sublinhado (comando \verb|\underline|): \underline{texto sublinhado};
% 	\item Negrito e itálico (usar comandos juntos): \textbf{\textit{texto em negrito e itálico}};
% 	\item Alterar cor do texto (comando \verb|\textcolor{cor}{texto}|):
% 	\begin{itemize}
% 		\item Exemplo \verb|\textcolor{red}{texto}|: \textcolor{red}{texto vermelho};
% 		\item Exemplo \verb|\textcolor[RGB]{255, 102, 0}|: \textcolor[RGB]{255, 102, 0}{texto laranja};
% 		\item Exemplo \verb|\textcolor[HTML]{006AD7}|: \textcolor[HTML]{006AD7}{texto azul};
% 	\end{itemize}
% 	\item Ambiente matemático inline (comando \verb|$ expressão $|): $s = x^2-2x +1$;
% 	\item Referência normal (comando \verb|\cite|):
% 	\begin{itemize}
% 		\item \cite{Agaisse1995};
% 		\item \cite{Abedi2014};
% 		\item \cite{BtNomenclature2016};
% 	\end{itemize}
% 	\item Referência normal com mais de uma obra (comando \verb|\cite|):
% 	\begin{itemize}
% 		\item \cite{Abedi2014, Agaisse1995};
%         \item \cite{AgapitoTenfen2014, BtNomenclature2016, Nelson2014};
% 	\end{itemize}
% 	\item Referência nome e ano (comando \verb|\citeauthorandyear|):
% 	\begin{itemize}
% 		\item \citeauthorandyear{Agaisse1995};
% 		\item \citeauthorandyear{Abedi2014};
% 		\item \citeauthorandyear{BtNomenclature2016};
% 	\end{itemize}
% \end{itemize}


% Exemplo 1 de citação direta:

% \begin{citacao}
% 	Os 20 aminoácidos usualmente encontrados como resíduos em proteínas contém um grupo $\alpha$-carboxil, um grupo $\alpha$-amino e um grupo R
% 	distinto substituído no átomo de carbono $\alpha$. O átomo de carbono $\alpha$ de todos os aminoácidos, com exceção da glicina, é assimétrico e,
% 	portanto, os aminoácidos podem existir em pelo menos duas formas estereoisoméricas. Somente os estereoisômeros L, com uma configuração relacionada à
% 	configuração absoluta da molécula de referência L-gliceraldeído, são encontrados em proteínas \cite[p. 81]{Nelson2014}.
% \end{citacao}

% Exemplo 2 de citação direta:

% \begin{citacao}
% 	\textit{These various insecticidal proteins are synthesized during the stationary phase and accumulate in the mother cell as a crystal inclusion
% 	which can account for up to 25\% of the dry weight of the sporulated cells. The amount of crystal protein produced by a B. thuringiensis culture in
% 	laboratory conditions (about 0.5 mg of protein per ml) and the size of the crystals (24) indicate that each cell has to synthesiz
% 	e $10^6$ to $2 \times 10^6$ $\delta$-endotoxin molecules during the stationary phase to form a crystal} \cite[p. 1]{Agaisse1995}.
% \end{citacao}

% Exemplo de nota de rodapé\footnote{Essa é uma nota de rodapé!}.


\section{Trabalhos Correlatos} % TODO: Perguntar ao professor.

Pesquise e descreva no mínimo três trabalhos correlatos ao seu.

\subsection{Interface Modo Texto x Modo Gráfico}

% 4 PÁGINAS (Duas páginas para cada modo? pensar)

% Deve conter: Histórico, exemplos...; Dar um gancho e falar dos editores de texto em modo texto (Vim, Emacs, Nano, helix, etc)
% e em modo gráfico (Notepad++, VSCode, Zed, etc); Falar que o foco do trabalho é editor  em modo texto.

Como os usuários interagiam com computadores antes das interfaces gráficas? Nos primórdios da computação, programas e dados eram armazenados em cartões
perfurados, estes que eram pré-processados e geravam saída após um longo tempo de espera. Com a constante evolução das maquinas e a criação dos mainframes, a
interação usuário-máquina passou a ser  feita principalmente por meio de interfaces de linha de comando (CLI), como o TTY. Nestes sistemas os usuários digitavam
comandos diretamente em um terminal e recebiam uma saída quase em tempo real.

Diferente da atualidade dominada pelas interfaces gráficas (GUI), as interfaces de linha de comando
exigiam que os usuários conhecessem os comandos exatos para realizar tarefas simples, como copiar arquivos ou executar programas. Essa ausência de uma interface
gráfica significava que os usuários precisavam memorizar uma série de comandos e sintaxes, o que tornava o uso do computador muito mais técnico e menos acessível
ao público em geral.

Um exemplo notável dessa era é o terminal VT100, lançado em 1978 pela Digital Equipment Corporation (DEC), pioneiro a introduzir códigos de escape ANSI e
permitindo controle absoluto do cursor e formatação de texto.

Ainda hoje, após decadas de evolução, ainda é possível encontrar usuários que preferem a interação por meio das CLIs, especialmente em ambientes
de desenvolvimento de sistemas. A popularidade de editores de texto em terminais, como o Vim em ambientes UNIX desmonstra exatamente a força dessa comunidade.


\subsection{Editores, Processadores de Texto e IDEs}

MEIA PÁGINA PARA CADA UM (ou mais se julgar necessário)

Deve conter: Listar editores antigos (por questões históricas); Talvez um detalhe baixo nível, sobre como funcionam os buffers de busca
em um texto.

\subsection{Bibliotecas para Lidar com Terminal} % TODO: Mudar nome depois

MEIA PÁGINA PARA CADA BIBLIOTECA

Deve conter: As principais bibliotecas para lidar com terminal (ncurses, termbox, etc); Falar um pouco sobre cada uma e
por que a escolhida foi a ncurses.
